%\documentclass{report}

\begin{document}
    \section{Cook-Levin Theorem}
        In this section
    
    \section{Kuratowski Theorem}
        In this section we prove [\ref{thm_kuratowski}].
        
        %(TODO: complete this, based on http://www.cs.xu.edu/~otero/math330/kuratowski.html http://www.cs.rpi.edu/~goldberg/14-GT/19-kurat.pdf and my lecture note)
    
    \subsection{The Preparation}
    
        First, we show that a planar graph can be drawn so that an arbitrary vertex or an edge is incident to the outer face.
    
        \begin{lemma} \label{lem_stereographic}
            If $G$ is planar and $v \in V(G)$, then there is a planar embedding of $G$ such that $v$ is on the boundary of the outer face. The same can be done for $e \in E(G)$.
        \end{lemma}
        
        \begin{proof}
            We use the \emph{stereographic projection}. In $\mathbb{R}^3$, let $z=-1$ be the plane $P$ and $x^2+y^2+z^2=1$ be the sphere $S$. $(0,0,1)$ is the ``north pole'' of $S$. Define the projection $\rho: S \backslash \{(0,0,1)\} \rightarrow P$ as follows: given $(x,y,z)$ on $S$ which is not the north pole, draw a straight line through $(0,0,1)$ and $(x,y,z)$. There is a unique intersection of this line with $P$, denoted as $(X,Y,-1)$. Then $\rho(x,y,z)=(X,Y,-1)$. Clearly $\rho$ is bijective.
            
            Given an embedding of a planar graph $G$ on $P$, $\rho^{-1}$ gives an embedding of $G$ on $S$. Rotate the embedding so that a face incident to $v$ or $e$ contains the north pole. $\rho$ gives an embedding of $G$ on $P$ such that the face is the outer face.
        \end{proof}
        
        Next, we introduce the notion of connectivity. Although connectivity is a crucial part of graph theory, we didn't put this into the main part of the codex because of the length concerns.
    
        \begin{defn}[Connectivity] \label{def_connectivity}
            A graph $G$ is \emph{$k$-connected} if $|V| > k$ and, for every $S \subset V$ with $|S| < k$, $G \backslash S$ is connected.
        \end{defn}
        
        \begin{thm} \label{thm_3conn}
            If $G$ is 3-connected with $|V(G)| \geq 5$, then there is an edge $e$ such that $G/e$ is 3-connected.
        \end{thm}
        
        \begin{proof}
            Let $e=xy$ and suppose $G/e$ is not 3-connected. Then $G/e$ has a cut set $\{v, z\}$. Since $G$ is 3-connected, this set has a vertex, say $v$, which is the new vertex made by contracting $e$. That is, $\{x, y, z\}$ is a cut set of $G$.
            
            Suppose that for every $e$, $G/e$ is not 3-connected, so to every $e$ corresponds a vertex $z_e$. Among all edges, take $e=xy$ and $z_e$ such that $G-x-y-z$ has the largest component $C$, and denote another component as $D$. Each of $x,y,z$ has neighbors in $C$ and in $D$ since $G$ is 3-connected. Take a neighbor $u$ of $z$ in $D$ and let $v=z_{zu}$.
            
            If $v \in V(C) \cup \{x,y\}$, then $G-z-v$ is disconnected, contradicting the connectivity of $G$. Otherwise, $G-z-u-v$ has a component that contains all vertices in $C$ and $x$ and $y$ in addition, contradicting the choice of $C$.
            
            (TODO: picture)
        \end{proof}
        
        Then, we show the connection between minors and topological minors.
        
        \begin{lemma} \label{lem_minor_K33}
            $K_{3,3}$ is a topological minor of $G$ iff $K_{3,3}$ is a minor of $G$.
        \end{lemma}
        
        \begin{proof}
            A topological minor of $G$ is also a minor of $G$. We just need to prove the other direction of the lemma.
            
            
        \end{proof}
        
        \begin{lemma} \label{lem_minor_K5}
            If $K_5$ is a minor of $G$, then $K_{3,3}$ or $K_5$ is a topological minor of $G$.
        \end{lemma}
        
        \begin{proof}
            .
        \end{proof}
    
    \subsection{The Proof}
    
        The last step is closely related to the Kuratowski's theorem.
    
        \begin{defn}[Convex Embedding] \label{def_convex_embed}
            A \emph{convex embedding} of a planar graph $G$ is a plane graph in which all edges are straight line segments and all face boundaries are convex polygons.
        \end{defn}
        
        \begin{lemma} \label{lem_convex_embed}
            If $G$ is simple, 3-connected, and has no $K_5$ or $K_{3,3}$ as a minor, then $G$ has a convex embedding on a plane, with no three vertices on a line.
        \end{lemma}
        
        \begin{proof}
            TODO
        \end{proof}
        
        We are finally ready to prove the Kuratowski's theorem. For convenience, we will restate the theorem:
        
        \begin{quote}
            A graph $G$ is planar if and only if it does not have $K_5$ or $K_{3,3}$ as a topological minor.
        \end{quote}
        
        \begin{proof}
            Induction on $|V|$, with trivial base case $|V| \leq 4$.
            
            If $G$ is disconnected, from induction there is a planar embedding of each component. Since each embedding is bounded by a finite disc, their union can be drawn on a plane.
            
            If $G$ is connected but not 2-connected, then take a cut-vertex $v$. Let $G_1$, $\cdots$, $G_n$ be the connected components of $G-v$, and $H_i$ be the subgraph induced by $V(G_i) \cup \{v\}$. Take an embedding of each $H_i$ such that $v$ is in the outer face [\ref{lem_stereographic}] and squeeze it into an angle $< 2\pi / n$ at the vertex $v$. Joining those embeddings together forms an embedding of $G$.
            
            If $G$ is 2-connected but not 3-connected, TODO
            
            If $G$ is 3-connected, the conclusion immediately follows from [\ref{lem_convex_embed}].
        \end{proof}
    
    \section{What's Wrong With Kempe's Proof?}
        
        Kempe argued that switching $V_{13}$ and $V_{52}$ allows $v$ to be colored by 1, but consider the following graph:
        
        (TODO: counterexample picture)
        
        Both chains cannot be switched because then the vertices $a$ and $b$ would have the same color!
        
        In this graph, such a problem could be avoided by deliberately changing the order of vertices to be selected for induction. However, there are graphs on which such a workaround is not possible. The following is the smallest counterexample possible, and is called the Soifer graph:
        
        (TODO: Soifer graph)
    
\end{document}