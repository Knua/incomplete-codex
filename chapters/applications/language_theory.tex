\documentclass{report}

\begin{document}
	Automaton is defined as a machine or control mechanism designed to automatically follow a predetermined sequence of operations, or respond to predetermined instructions. Theoretically, they all can be considered as the simplest form of algorithm, whether it is finite state automaton, push down automaton, or Turing machine. They all accept an input, and produce output; usually the output is $accept$ or $reject$, but in the case of Turing machines, the output may be something different.
	
	Before we start talking about the machines however we need to define what "Language" is.
	\begin{defn}[Language] \label{def_language}
		A (formal) language $L$ over an alphabet $\Sigma$ is a subset of $\Sigma^*$, that is, a set of words over that alphabet.
	\end{defn}
	
	In this section, we explore \emph{Regular Language}[\ref{def_RL}], \emph{Context-free Language}[\ref{def_cfl}], \emph{Decidable Language}[\ref{def_decidable_language}], and \emph{Recognizable Language}[\ref{def_recognizable_language}] and the mechanisms, or machines, that are related those languages.
	
	\section{Regular Language}
		\subsection{Regular Expression}
			If you have studied regular expression using some programming languages, then you might have easier time understanding the following definition. The regular expression used in real life is much more powerful than the regular expression mentioned below, as more special characters and syntaxes are allowed. However the following regular expression consists of the "basics" of regular expression, and is used in language theories as the regular expression:
			\begin{defn}[Regular Expression(RE)] \label{def_RE}
				Given a finite alphabet $\Sigma$, the following constants are defined as regular expressions:
				\begin{itemize}
					\item \textbf{Empty set}: $\emptyset$, denoting the set $\emptyset$.
					\item \textbf{Empty string}: $\epsilon$, denoting the set containing only the "empty" string, which has no characters at all.
					\item \textbf{Literal character}: $a \in \Sigma$, denoting the only character $a$.
				\end{itemize}
				And when given regular expressions $R$ and $S$, the following operations over them produce regular expressions:
				\begin{itemize}
					\item \textbf{Concatenation}: $RS$, denoting the concatenation of strings in $R$ and $S$, in that order.
						\subitem $R^n$ denotes the concatenation of $R$, $n$ times: Specifically, $R^0=\{\epsilon\}$.
					\item \textbf{Alternation}: $R|S$, denoting the set union of the strings in $R$ and $S$.
					\item \textbf{Kleene star}: $R^*$, denoting $\bigcup_{i \in \mathbb{N}}R^i$.
				\end{itemize}
			\end{defn}
			
			\begin{defn}[Regular Languages] \label{def_RL}
				\emph{Regular Languages} are languages that can be represented with regular expressions.
			\end{defn}
			
			\begin{thm}[Pumping Lemma for Regular Languages] \label{thm_pumping_lemma_RL}
				Let $L$ be a regular language. Then, there exists an integer $p \ge 1$, depending only on $L$, such that every string $w \in L$ of length at least $p$, called the \emph{pumping length}, can be written as $w=xyz$(i.e. $w$ can be divided into three substrings), satisfying the following conditions:
				\begin{itemize}
					\item $|y| \ge 1$
					\item $|xy| \le p$
					\item $\forall n \ge 0, xy^nz \in L$
				\end{itemize}
			\end{thm}
		
		\subsection{Deterministic Finite State Automaton}
			\begin{defn}[Deterministic Finite Automaton(DFA)] \label{def_DFA}
				A DFA $M$ is a 5-tuple, $(Q,\Sigma,\delta,q_0,F)$, consisting of:
				\begin{itemize}
					\item Finite set of states $Q$;
					\item Finite set of input symbols called the alphabet $\Sigma$;
					\item Transition function $\delta:Q \times \Sigma \rightarrow Q$;
					\item Initial state $q_0 \in Q$;
					\item Set of accepting states $F \subseteq Q$.
				\end{itemize}
				
				Let $w=a_1a_2\dots a_n$ be a string over the alphabet $\Sigma$. DFA $M$ accepts the string $w$ if a sequence of states, $r_0,r_1,\dots,r_n \in Q$ exists with the following conditions:
				\begin{itemize}
					\item $r_0=q_0$
					\item $r_{i+1}=\delta (r_i,a_{i+1}), i=0,\dots,n-1$
					\item $r_n \in F$
				\end{itemize}
			\end{defn}
		
			\begin{thm} \label{thm_DFA_RL}
				DFAs recognize exactly the set of regular languages.
			\end{thm}
		
		\subsection{Nondeterministic Finite Automaton}
			\begin{defn}[Nondeterministic Finite Automaton(NFA)] \label{def_NFA}
				A NFA $M$ is a 5-tuple, $(Q,\Sigma,\Delta,q_0,F)$, consisting of:
				\begin{itemize}
					\item Finite set of states $Q$;
					\item Finite set of input symbols called the alphabet $\Sigma$;
					\item Transition function $\Delta: Q \times \Sigma \rightarrow P(Q)$ where $P$ is the powerset function;
					\item Initial state $q_0 \in Q$;
					\item Set of accepting states $F \subseteq Q$.
				\end{itemize}
				Sometimes the transition function $\Delta$ is represented as the transition relation, $\Delta \subseteq Q \times \Sigma \times Q$.
				
				Let $w=a_1a_2\dots a_m$ be a string over the alphabet $\Sigma$, where $a_i \in \Sigma_\epsilon$. NFA $M$ accepts the string $w$ if a sequence of states, $r_0,r_1,\dots,r_n \in Q$ exists with the following conditions:
				\begin{itemize}
					\item $r_0=q_0$
					\item $r_{i+1}\in \Delta (r_i,a_{i+1})$, or in relation form, $(r_i,a_{i+1},r_{i+1}) \in \Delta$, $i=0,\dots,n-1$
					\item $r_n \in F$
				\end{itemize}
			\end{defn}
			
			\begin{thm} \label{thm_NFA_RL}
				NFAs recognize exactly the set of regular languages.
			\end{thm}
	
	\section{Context-Free Language}
		\subsection{Context-free Grammar}
			\begin{defn}[Context-Free Grammar(CFG)] \label{def_cfl}
				A CFL is a 4-tuple $(V,\Sigma,R,S)$ where:
				\begin{itemize}
					\item $V$ is the set of nonterminal variables;
					\item $\Sigma$ is the set of terminal characters;
					\item $R$ is the set of rules, where each rules are in the form of $A\rightarrow w, A\in V, w\in (\Sigma \cup V)^*$
					\item $S$ is the starting variable.
				\end{itemize}
			\end{defn}
			
			\begin{defn}[Context-free Languages] \label{def_CFL}
				\emph{Context-free Languages} are languages that can be represented with context-free grammars.
			\end{defn}
			
			\begin{thm}[Pumping Lemma for Context-free Languages] \label{thm_pumping_lemma_CFL}
				Let $L$ be a regular language. Then, there exists an integer $p \ge 1$, depending only on $L$, such that every string $s \in L$ of length at least $p$, called the \emph{pumping length}, can be written as $s=uvwxy$(i.e. $w$ can be divided into five substrings), satisfying the following conditions:
				\begin{itemize}
					\item $|vx| \ge 1$
					\item $|vwx| \le p$
					\item $\forall n \ge 0, uv^nwx^ny \in L$
				\end{itemize}
			\end{thm}
		\subsection{Push-down Automaton}
			Similar to Finite Automatons, Push-down automatons have deterministic version and nondeterministic version; Only the nondeterministic version is shown here as similar method can be used to convert it into a deterministic version.
			\begin{defn}[Push-down Automaton(PDA)] \label{def_pda}
				A PDA is a 6-tuple $(Q,\Sigma,\Gamma,q_0,\Delta,F)$ where:
				\begin{itemize}
					\item $Q$ is the set of states;
					\item $\Sigma$ is the set of input alphabet;
					\item $\Gamma$ is the set of stack alphabet;
					\item $q_0$ is the starting state;
					\item $\Delta$ is the transition relation of $Q \times \Sigma_\epsilon \times \Gamma_\epsilon \times Q \times \Gamma_\epsilon$
					\item $F$ is the set of accepting states
				\end{itemize}
				$\Delta$ is often written as a transition function of $Q \times \Sigma_\epsilon \times \Gamma_\epsilon \times \rightarrow P(Q \times \Gamma_\epsilon)$ where $P$ is the powerset function.\\
				Sometimes the last element of the relation is extended to $\Gamma_\epsilon^*$; in that case, when inserting into the stack, insert the last element first.
				Let $w=w_1w_2\dots w_m$ be a string over the alphabet $\Sigma$, where $w_i \in \Sigma_\epsilon$. NFA $M$ accepts the string $w$ if a sequence of states, $r_0,r_1,\dots,r_n \in Q$, and a sequence of stack strings $s_0,s_1,\dots,s_n \in \Gamma^*$ exists with the following conditions:
				\begin{itemize}
					\item $r_0=q_0$, $s_0=\epsilon$
					\item $(r_i,w_{i+1},a,r_{i+1},b)\in \Delta$, where $s_i=at$ and $s_{i+1}=bt$ for some $a,b \in \Gamma_\epsilon$, and $t \in \Gamma^*$.
					\subitem If $b=\epsilon$, then it is a pop-operation. If $a=\epsilon$, then it is a push-operation.
					\item $r_m \in F$, $s_m=\epsilon$
				\end{itemize}
			\end{defn}
			
			\begin{thm} \label{thm_PDA_CFL}
				PDAs recognize exactly the set of CFLs.
			\end{thm}
		
			\begin{proof}
				The proof is quite tedious; so only a partial proof is given: we are going to convert any given CFG into PDA.
				Suppose a CFG $(V_0,\Sigma_0,R_0,S_0)$ is given.\\
				We can construct a new PDA $(Q,\Sigma,\Gamma,q_0,\Delta,F)$ from the given CFL s.t.
				\begin{itemize}
					\item $Q=\{Q_S,Q_M,Q_F\}$
					\item $\Sigma=\Sigma_0$
					\item $\Gamma=V_0 \cup \Sigma_0$
					\item $q_0=Q_S$
					\item $F=Q_F$
					\item $\Delta=$\\
					\begin{math}
					\{(Q_S,\epsilon,\epsilon,Q_M,S\$)\}\cup\\
					\{(Q_M,\epsilon,\epsilon,X,Q_M,W)|X->W \in R\}\cup\\
					\{(Q_M,a,a,Q_M,\epsilon)|a \in \Sigma_0\}\cup\\
					\{(Q_M,\epsilon,\$,Q_F,\epsilon)\}
					\end{math}
				\end{itemize}
				This exactly simulates the parse tree of the CFL.
			\end{proof}
	
	\section{Turing Machines}
		\begin{defn}[Turing Machine] \label{def_TM}
			A TM is a 7 tuple $(Q,\Sigma,\Gamma,\delta,q_0,q_{accept},q_{reject})$ where:
			\begin{itemize}
				\item $Q$ is the set of states;
				\item $\Gamma$ is the set of tape alphabet;
				\item $b \in \Gamma$ is the blank symbol, the only symbol allowed to occur infinitely often at any step of the computation;
				\item $\Sigma \subseteq \Gamma \textbackslash \{b\}$ is the set of input symbols, that is, the set of symbols allowed to appear in the initial tape contents;
				\item $q_0 \in Q$ is the starting state;
				\item $F \subseteq Q$ is the set of accepting states, and the initial tape contents is said to be accepted by $M$ if it eventually halts in a state from $F$;
				\item $\delta$ is a partial function called the transition function of $(Q \textbackslash F) \times \gamma \rightarrow Q \times \gamma \times \{L,R\}$, where $L$ and $R$ signifies left and right shifts of the tape. If $\delta$ is undefined on the current state and the current tape symbol, then the machine halts.
			\end{itemize}
			A Turing machine 
		\end{defn}
		The definition of a Turing Machine is not unique. Some definitions use multiple tapes, using one of them as the input tape that can't be modified and another as the output tape. Some has more than one halting states. Some include the "starting symbol" in the alphabet. Some include $N$ in the final output of the transition function. But in general, a Turing machine starts from one state, follows the decision function every step, and halts at the halting state. Some of the many variations on the Turing machine are mentioned in \ref{var_TM}.
		
		In fact, the different definitions of a Turing machine turns out to be the same, in the sense that a function $f:\strs \rightarrow \zo$ is computable using one definition of a Turing machine iff it is computable using another definition of a Turing Machine.
		
		We now give the following thesis from the creator of the $\lambda$-calculus, Alonzo Church and Alan Turing.
		\begin{thesis}[Church-Turing Thesis] \label{thm_church_turing}
			A function on Natural Numbers which is computable by a human being following an algorithm, ignoring resource limitations, if and only if it is computable by a Turing Machine.
		\end{thesis}
		
		\section{Decidable and Recognizable Languages}
		\begin{defn}[Decidable Languages] \label{def_decidable_language}
			\emph{Decidable Languages} are languages that can be represented with decidable Turing machines; that is, the set of Turing machines that always accepts accepting words and rejects others.
		\end{defn}
	
		\begin{defn}[Recognizable Languages] \label{def_recognizable_language}
			\emph{Recognizable Languages} are languages that can be represented with recognizable Turing machines; that is, the set of Turing machines that always accepts accepting words.
		\end{defn}
		
		Decidable and Recognizable Turing Machines seem similar; however recognizable machines does not have to reject a non-accepting word; it may instead loop infinitely.
		
		\section{Equivalences to Turing Machine}
			The followings can be shown to be computationally equivalent to a Turing machine; however no proofs are given since they are usually long and arduous.

			\subsection{Push-down Automaton with Two Stacks}
			The simplest version that is equivalent to a Turing Machine would be a PDA which has two stacks. The two stacks can simulate the tape of the Turing machine by pushing and popping.
			
			\subsection{Variations on the Turing Machine} \label{var_TM}
			The following variations on the Turing machine are equivalent to the original Turing machine:
			\begin{itemize}
				\item Variations on the Definition
				\begin{itemize}
					\item Allowing $N$, "no shift", in the movement rules;
					\item Having a single accepting state, say $q_{accept}$ and a single rejecting state, say $q_{reject}$, and forcing the transition function $\delta$ to be a function. In this variant, the machine accepts iff it ends in $q_{accept}$, and rejects iff it ends in $q_{reject}$.
				\end{itemize}
				\item Variations on the Form of the Machine
				\begin{itemize}
					\item Tape is infinite only in one direction;
					\item Tape is infinite in both directions;
					\item Tape is 2-dimensional;
					\item There exists multiple tapes that the machine can access concurrently.
				\end{itemize}
			\end{itemize}
			There are many more variations other than these.

			\subsection{General Recursive Functions}
				\begin{defn}[General Recursive Functions] \label{def_general_recursive_function}
					General Recursive Functions, otherwise known as $\mu$-recursive functions, is a set of functions $\forall n \in \mathbb{N}, f:\mathbb{N}^n \rightarrow \mathbb{N}$ that includes the three "Initial", or "Basic" functions, and closed under three operators:
					\begin{itemize}
						\item Initial Functions
						\begin{itemize}
							\item \textbf{Constant Function:} $\forall n,k \in \mathbb{N}, f(x_1,\dots,x_k)=n$
								\subitem Alternative definition use a Zero function: $\forall k \in \mathbb{N}, Z(x_1,\dots,x_k)=0$
							\item \textbf{Successor Function $S$:} $S(x)=x+1$
							\item \textbf{Projection Function $P_i^k$:}
								\subitem This is also called the Identity Function $I_i^k$
						\end{itemize}
						\item Operators
						\begin{itemize}
							\item \textbf{Composition Operator $\circ$:} Given an m-ary function $h(x_1,\dots,x_m)$ and m k-ary functions $g_1(x_1,\dots,x_k),\dots,g_m(x_1,\dots,x_k)$:
							\begin{displaymath}
								h \circ (g_1,\dots,g_m)=f \text{   where   } f(x_1,\dots,x_k)=h(g_1(x_1,\dots,x_k),\dots,g_m(x_1,\dots,x_k))
							\end{displaymath}
								\subitem This is also called the Substitution Operator.
							\item \textbf{Primitive Recursion Operator $\rho$:} Given the k-ary function $g(x_1,\dots,x_k)$ and (k+2)-ary function $h(y,z,x_1,\dots,x_k)$:
							\begin{align*}
								\rho(g,h)&=f \text{   where}\\
								f(0,x_1,\dots,x_k)&=g(x_1,\dots,x_k)\\
								f(y+1,x_1,\dots,x_k)&=h(y,f(y,x_1,\dots,x_k),x_1,\dots,x_k)
							\end{align*}
							\item \textbf{Minimization Operator $\mu$:} Given a (k+1)-ary total function $f(y,x_1,\dots,x_k)$:
							\begin{align*}
								\mu (f)(x_1,\dots,x_k)=z \Leftrightarrow& f(z,x_1,\dots,x_k)=0 \text{   and}\\&f(i,x_1,\dots,x_k)>0 \text{   for   } i=0,\dots,z-1
							\end{align*}
								\subitem Intuitively, this operator seeks the smallest argument that causes the function to return 0; if none exists, the search never ends and therefore cannot return.
						\end{itemize}
					\end{itemize}
				\end{defn}
			
			\subsection{Lambda Calculus}
				Lambda Calculus, first defined by Alonzo Church, is a formal system of mathematical logic for expressing computation based on function-like objects.
				\begin{defn}[Lambda Expression] \label{def_lambda_expression}
					Lambda expressions are composed of:
					\begin{itemize}
						\item Variables, $v_1,\dots,v_n,\dots$
						\item The abstraction symbols lambda $\lambda$ and dot $.$
						\item Parentheses $( )$
					\end{itemize}
					For some applications, terms for logical and mathematical constants and operation may be included.
					
					The set of lambda expressions, $\Lambda$, can be defined inductively:
					\begin{itemize}
						\item If $x$ is a variable, then $x \in \Lambda$
						\item If $x$ is a variable and $M \in \Lambda$, then $(\lambda x.M)\in \Lambda$
							\subitem This rule is also known as Abstractions.
						\item If $M,N \in \Lambda$, then $(M N) \in \Lambda$
							\subitem This rule is also known as Application.
					\end{itemize}
				\end{defn}
				Though only the definition is given, \href{https://en.wikipedia.org/wiki/Lambda_calculus#Formal_definition}{This Wikipedia article} can be helpful to understand how lambda calculus works.
			
\end{document}