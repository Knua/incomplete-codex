\documentclass{report}

\begin{document}

	\section{Algebraic Structures}

		\subsection{Sets}

			\begin{defn}[Set] \label{def_set}
				A \emph{set} is a collection of distinct objects. 
			\end{defn}

			\begin{defn}[Order] \label{def_order}
				Let $S$ be a set. An \emph{order} on $S$ is a relation, denoted by $<$, with the following properties:
				\begin{itemize}
					\item If $x \in S$ and $y \in S$ then one and only one of the following statements is true:
					\begin{displaymath}
						x<y, x=y, y<x
					\end{displaymath}
					\item For $x,y,z \in S$, if $x<y$ and $y<z$, then $x<z$.
				\end{itemize}
			\end{defn}

			\begin{remark} \label{remark_order}
				\begin{itemize}
					\item[]
					\item It is possible to write $x>y$ in place of $y<x$
					\item The notation $x \leq y$ indicates that $x<y$ or $x=y$.
				\end{itemize}
			\end{remark}

		\begin{defn}[Ordered Set] \label{def_ordered_set}
			An \emph{ordered set} is a set in which an order is defined.
		\end{defn}

		\begin{defn}[Bound] \label{def_bound}
			Suppose $S$ is an ordered set, and $E\subset S$.\\
			If there exists $\beta \in S$ such that $x \leq \beta$ for every $x \in E$, we say that E is \emph{bounded above}, and call $\beta$ an \emph{upper bound} of E.
			If there exists $\alpha \in S$ such that $x \geq \alpha$ for every $x \in E$, we say that E is \emph{bounded below}, and call $\alpha$ a \emph{lower bound} of E.
		\end{defn}

		\begin{defn}[Least Upper Bound] \label{def_supremum}
			Suppose that $S$ is an ordered set, and $E \subset S$.
			If there exists a $\beta \in S$ with the following properties:
			\begin{itemize}
				\item $\beta$ is an upper bound of $E$
				\item If $\gamma < \beta$, then $\gamma$ is not an upper bound of E
			\end{itemize}
			Then $\beta$ is called the \emph{Least Upper Bound} of E or the \emph{supremum} of E, denoted
			\begin{displaymath}
				\beta=sup(E)
			\end{displaymath}
		\end{defn}

		\begin{defn}[Greatest Lower Bound] \label{def_infimum}
			Suppose that $S$ is an ordered set, and $E \subset S$.
			If there exists a $\alpha \in S$ with the following properties:
			\begin{itemize}
				\item $\alpha$ is a lower bound of $E$
				\item If $\gamma < \alpha$, then $\gamma$ is not an lower bound of E
			\end{itemize}
			Then $\alpha$ is called the \emph{Greatest Lower Bound} of E or the \emph{infimum} of E, denoted
			\begin{displaymath}
			\beta=inf(E)
			\end{displaymath}
		\end{defn}

		\begin{defn}[least-upper-bound property] \label{def_least_upper_bound_property}
			An ordered set $S$ is said to have the \emph{least-upper-bound property} if the following is true:\\
			if $E \subset S$, $E$ is not empty, and $E$ is bounded above, then $sup(E)$ exists in $S$.
		\end{defn}

		\begin{defn}[greatest-lower-bound property] \label{def_greatest_lower_bound_property}
			An ordered set $S$ is said to have the \emph{greatest-lower-bound property} if the following is true:\\
			if $E \subset S$, $E$ is not empty, and $E$ is bounded below, then $inf(E)$ exists in $S$.
		\end{defn}

		\begin{thm} \label{thm_glb_lub_property}
			Suppose $S$ is an ordered set with the least-upper-bound property, $B \subset S$, $B$ is not empty, and $B$ is bounded below.\\
			Let $L$ be the set of all lower bounds of $B$. Then
			\begin{displaymath}
				\alpha=sup(L)
			\end{displaymath}
			exists in $S$, and $\alpha=inf(B)$.
		\end{thm}

		\begin{proof}
			Note that $\forall x \in L, y \in B, x \leq y$.\\
			$L$ is nonempty as $B$ is bounded below.\\
			$L$ is bounded above since $\forall x \in S \backslash L, \forall y \in L, x>y$.\\
			Since $S$ has the least-upper-bound property and $L \subset S$, $\exists \alpha=sup(L)$.\\
			The followings hold:
			\begin{itemize}
				\item $\alpha$ is a lower bound of $B$.
					\\($\because$) $\forall \gamma \in B, \gamma > \alpha$
				\item $\beta$ with $\beta > \alpha$ is not a lower bound of $B$
					\\($\because$)Since $\alpha$ is an upper bound of $L$, $\beta \notin L$.
			\end{itemize}
			Hence $\alpha=inf(B)$.
		\end{proof}
		
		\begin{coro} \label{coro_glb_lub_property_equiv}
			For all ordered sets, the Least Upper Bound property and the Greatest Lower Bound Porperty are equivalent.
		\end{coro}

		\subsection{Group}

			\begin{defn}[Group] \label{def_group}
				A \emph{group} is a set $G$ with a binary operation $\cdot$, denoted $(G,\cdot)$, which satisfies the following conditions:
				\begin{itemize}
					\item \textbf{Closure}: $\forall a,b \in G, a \cdot b \in G$
					\item \textbf{Associativity}: $\forall a,b,c \in G, (a \cdot b) \cdot c=a \cdot (b \cdot c)$
					\item \textbf{Identity}: $\exists e \in G, \forall a \in G, a \cdot e=e \cdot a=a$
					\item \textbf{Inverse}: $\forall a \in G, \exists a^{-1} \in G, a \cdot a^{-1}=a^{-1} \cdot a=e$
				\end{itemize}
			\end{defn}

		\begin{defn}[Semigroup] \label{def_semigroup}
			A \emph{semigroup} is $(G,\cdot)$, which satisfies Closure and Associativity.
		\end{defn}

		\begin{defn}[Monoid] \label{def_monoid}
			A \emph{monoid} is a semigroup $(G,\cdot)$ which also has identity.
		\end{defn}

		\begin{defn}[Abelian Group] \label{def_abelian_group}
			An \emph{Abelian Group} or \emph{Commutative Group} is a group $(G,\cdot)$ with the following property:
			\begin{itemize}
				\item \textbf{Commutativity}: $\forall a,b \in G, a \cdot b=b \cdot a$
			\end{itemize}
		\end{defn}

		\subsection{Ring}

			\begin{defn}[Ring] \label{def_ring}
				A \emph{Ring} is a set $R$ with two binary operations $+$ and $\cdot$, often called the addition and multiplication of the ring, denoted $(R,+,\cdot)$, which satisfies the following conditions:
				\begin{itemize}
					\item $(R,+)$ is an abelian group
					\item $(R,\cdot)$ is a semigroup
					\item \textbf{Distribution}: $\cdot$ is distributive with respect to $+$, that is, $\forall a,b,c \in R$:
					\begin{itemize}[label=-]
						\item $a \cdot (b + c)=(a \cdot b) + (a \cdot c)$
						\item $(a + b) \cdot c=(a \cdot c) + (b \cdot c)$
					\end{itemize}
				\end{itemize}
			The identity element of $+$ is often noted $0$.
			\end{defn}

		\begin{defn}[Ring with identity(1)] \label{def_ring_with_1}
			A \emph{Ring with identity} is a ring $(R,+,\cdot)$ of which $(R,\cdot)$ is a monoid. The identity element of $\cdot$ is often noted $1$.
		\end{defn}

		\begin{defn}[Commutative Ring] \label{def_commutative_ring}
			A \emph{commutative ring} is a ring $(R,+,\cdot)$ of which $\cdot$ is commutative.
		\end{defn}

		\begin{defn}[Zero Divisor] \label{def_zero_divisor}
			For a ring $(R,+,\cdot)$, let $0$ be the identity of $+$.\\
			$a,b\in R$, $a \neq 0$ and $b \neq 0$, if $a \cdot b=0$, $a,b$ are called the zero divisors of the ring.
		\end{defn}

		\begin{defn}[Integral Domain] \label{def_integral_domain}
			An \emph{integral domain} is a commutative ring $(R,+,\cdot)$ with 1 which does not have zero divisors.
		\end{defn}

		\subsection{Field}

		\begin{defn}[Field] \label{def_field}
			A \emph{Field} is a set $F$ with two binary operations $+$ and $\cdot$, often called the addition and multiplication of the field, denoted $(R,+,\cdot)$, which satisfies the following conditions:
			\begin{itemize}
				\item $(F,+,\cdot)$ is a ring
				\item $(F\backslash \{0\},\cdot)$ is a group
			\end{itemize}
			Alternatively, a Field may be defined with a set of \emph{Field Axioms} listed below:
			\begin{itemize}
					\item[(A)] \textbf{Axioms for Addition}
					\begin{itemize}
						\item[(A1)] \textbf{Closed under Addition}\\$\forall a,b \in F, a+b \in F$
						\item[(A2)] \textbf{Addition is Commutative}\\$\forall a,b \in F, a+b=b+a$
						\item[(A3)] \textbf{Addition is Associative}\\$\forall a,b,c \in F, (a+b)+c=a+(b+c)$
						\item[(A4)] \textbf{Identity of Addition}\\$\exists 0 \in F, \forall a \in F, 0+a=a$
						\item[(A5)] \textbf{Inverse of Addition}\\$\forall a \in F, \exists -a \in F, a+(-a)=0$
					\end{itemize}
					\item[(M)] \textbf{Axioms for Multiplication}
					\begin{itemize}
						\item[(M1)] \textbf{Closed under Multiplication}\\$\forall a,b \in F, a \cdot b \in F$
						\item[(M2)] \textbf{Multiplication is Commutative}\\$\forall a,b \in F, a \cdot b=b \cdot a$
						\item[(M3)] \textbf{Multiplication is Associative}\\$\forall a,b,c \in F, (a \cdot b) \cdot c=a \cdot (b \cdot c)$
						\item[(M4)] \textbf{Identity of Multiplication}\\$\exists 1 \in F, \forall a \in F, 1 \cdot a=a$
						\item[(M5)] \textbf{Inverse of Multiplication}\\$\forall a \in F\backslash\{0\}, \exists a^{-1} \in F, a \cdot a^{-1}=1$
					\end{itemize}
				\item[(D)] \textbf{Distributive Law}
				\\$\forall a,b,c \in F, (a+b) \cdot c=a \cdot c+b \cdot c$\\where $\cdot$ takes precedence over $+$.
			\end{itemize}
		\end{defn}

		\begin{defn}[Ordered Field] \label{def_ordered_field}
			An \emph{ordered field} is a field $F$ which is an ordered set, such that
			\begin{itemize}
				\item $x+y<x+z$ if $x,y,z \in F$ and $y<z$
				\item $xy>0$ if $x,y \in F$, $x>0$ and $y>0$
			\end{itemize}
			\end{defn}

		\begin{thm}[Existence of $\mathbb{R}$] \label{thm_existence_real_number}
			There exists an ordered field $\mathbb{R}$ containing $\mathbb{Q}$ as a subfield which has the least-upper-bound property.
		\end{thm}

		\begin{defn}[Extended Real Number System] \label{def_extended_real_number_system}
			The \emph{extended real number system}, denoted $\overline{\mathbb{R}}$, $[-\infty,\infty]$, or $\mathbb{R} \cup \{-\infty,\infty\}$, consists of the real field $\mathbb{R}$ and two symbols, $+\infty$ and $-\infty$. We preserve the original order in $\mathbb{R}$, and define $\forall x \in \mathbb{R}$,
			\begin{displaymath}
				-\infty<x<\infty
			\end{displaymath}
		\end{defn}

		\begin{remark} \label{remark_extended_real_number_system_not_field}
			The extended real number system does not form a field.
		\end{remark}
	
		\subsection{Polynomial Ring}
		\begin{defn}[Polynomial over a Ring] \label{def_polynomial}
			A polynomial $f(x)$ over the ring $(R,+,\cdot)$ is defined as
			\begin{displaymath}
				f(x)=\sum_{i=0}^{\infty}a_ix^i=a_0+a_1x^1+\cdots,a_i\in R
			\end{displaymath}
			where $a_i=0$ for all but finitely many values of $i$.\\
			The \emph{degree} of the polynomial $\deg(f)$ is defined as $\deg(f)=\max\{n|n\in\mathbb{N}, a_n\ne0 \}$.
			The \emph{leading coefficient} of the polynomial is defined as $a_{\deg(f)}$.
		\end{defn}
		\begin{defn}[Addition and Multiplication of Polynomials] \label{def_add_mult_polynomial}
			Let $f(x)=\sum_{i=0}^{\infty}a_ix^i$, $g(x)=\sum_{i=0}^{\infty}b_ix^i$, $a_i,b_i \in R$ be a polynomial over the ring $(R,+,\cdot)$. Define:
			\begin{gather*}
				f(x)+g(x)=\sum_{i=0}^{\infty}(a_i+b_i)x^i\\
				f(x)g(x)=\sum_{k=0}^{\infty}(c_k)x^k \text{ where } c_k=\sum_{i+j=k}a_ib_j
			\end{gather*}
		\end{defn}
		\begin{defn}[Polynomial Ring] \label{def_polynomial_ring}
			The set of polynomials over the ring $(R,+,\cdot)$, $R[x]=\{f(x)|f(x) \text{ is a polynomial over } R \}$ is called the \emph{Polynomial Ring(or Polynomials) over $R$}.
		\end{defn}
		\begin{thm}[Degree of Polynomial on Addition and Multiplication] \label{thm_add_mult_deg}
			Let $f(x),g(x) \in R[x]$ with $\deg(f)=n$, $\deg(g)=m$.
			\begin{itemize}
				\item $0 \le \deg(f+g) \le \max(\deg(f),\deg(g))$
				\item $\deg(fg) \le \deg(f)+\deg(g)$.
				\subitem If $(R,+,\cdot)$ is an integral domain, $\deg(fg) = \deg(f)+\deg(g)$
			\end{itemize}
		\end{thm}
		\begin{thm}[Relationship between a Ring and its Polynomial Ring] \label{thm_ring_polynomial_relationship}
			Let $(R,+,\cdot)$ be a ring and $R[x]$ the polynomials over $R$.
			\begin{enumerate}
				\item If $(R,+,\cdot)$ is a commutative ring with $1$, then $(R[x],+,\cdot)$ is a commutative ring with $1$.
				\item If $(R,+,\cdot)$ is a integral domain, then $(R[x],+,\cdot)$ is a integral domain.
			\end{enumerate}
		\end{thm}
		\begin{thm}[Division Algorithm for Polynomials over a Ring] \label{thm_polynomial_division_algorithm_ring}
			Let $(R,+,\cdot)$ be a commutative ring with $1$.\\
			Let $f(x),g(x) \in R[x]$, $g(x) \ne 0$ with the leading coefficient of $g(x)$ being invertible.\\
			Then, $\exists! q(x),r(x) \in R[x]$ such that
			\begin{displaymath}
				f(x)=q(x)g(x)+r(x)
			\end{displaymath}
			where either $r(x)=0$ or $\deg(r)<\deg(g)$.
		\end{thm}
		\begin{proof}
				Use induction on $\deg(f)$.\\
				1. $f(x)=0$ or $\deg(f)<\deg(g)$: $q(x)=0, r(x)=f(x)$\\
				2. $\deg(f)=\deg(g)=0$: $q(x)=f(x) \cdot g(x)^{-1}, r(x)=0$\\
				3. $\deg(f)\ge\deg(g)$:\\
				
				1) Existence\\
				Let $\deg(f)=n$, $\deg(g)=m$, $n>m$.\\
				Suppose the theorem holds for $\deg(f)<n$.\\
				Let $f(x)=a_0+a_1x^1+\cdots+a_nx^n$, $g(x)=b_0+b_1x^1+\cdots+b_mx^m$.\\
				Choose $f_1(x)=f(x)-(a_nb_m^{-1})x^{n-m}g(x)\in R[x]$.\\
				Since $\deg(f_1)<n$, $\exists q(x),r(x)\in R[x]$ so that $f_1(x)=g(x)q(x)+r(x)$, where $r(x)=0$ or $\deg(r)<\deg(g)$.\\
				$f_1(x)=f(x)-(a_nb_m^{-1})x^{n-m}g(x)=g(x)q(x)+r(x)$\\
				$f(x)=g(x)((a_nb_m^{-1})x^{n-m}+q(x))+r(x)$\\
				Hence such pair exists.\\
				
				2) Uniqueness\\
				Suppose $f(x)=g(x)q_1(x)+r_1(x)=g(x)q_2(x)+r_2(x)$.\\
				$g(x)(q_1(x)-q_2(x))=r_2(x)-r_1(x)$\\
				If $r_1 \ne r_2$, $\deg(g)>\deg(r_2-r_1)=\deg(g(q_1-q_2))$.\\
				Since $\deg(g(q_1-q_2))\ge\deg(g)$ if $q_1-q_2\ne0$, $q_1=q_2$, but if so, $r_1=r_2$.\\
				If $r_1=r_2$, trivially $q_1=q_2$.\\
				Hence they exist uniquely.
		\end{proof}
	\section{From $\mathbb{N}$ to $\mathbb{R}$}

\end{document}