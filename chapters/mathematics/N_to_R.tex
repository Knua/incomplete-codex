\documentclass{report}

\begin{document}
	\section{$\mathbb{N}$: The set of Natural Numbers}
		\subsection{Construction of $\mathbb{N}$}
		We start from the Axioms of Set and the following definitions:
		
		\begin{defn}[Successor] \label{defn_successor}
			For any set $x$, the \emph{successor of $x$}, denoted $\sigma(x)$, is defined as the following set:
			\begin{displaymath}
			\sigma(x)=x \cup \{x\}
			\end{displaymath}
		\end{defn}
		
		Let us define $0=\emptyset$, $1=\sigma(\emptyset)=\sigma(0)$. Using the definition of successors, and following the pattern, $2=\sigma(1)$, $3=\sigma(2)$, and so on. Basically we can make any finite number using the definition of successor and the Axioms of Set, but actually getting all of the natural numbers at once(or any infinitely large set, since only the empty set is guaranteed to exist by the axioms) is not possible with our axioms. We define the concept of Inductive Sets and make another Axiom for this purpose:
		
		\begin{defn}[Inductive Set] \label{defn_inductive_set}
			A set $A$ is called \emph{inductive} if it satisfies the following two properties:
			\begin{itemize}
				\item $\emptyset \in A$
				\item $(x \in A) \Rightarrow (\sigma(x) \in A)$
			\end{itemize}
		\end{defn}
		
		\begin{axiom}[Axiom of Infinity] \label{axiom_infinity}
			There is an inductive set, that is:
			\begin{displaymath}
			\exists A (\emptyset \in A) \wedge (\forall x \in A, \sigma(x) \in A)
			\end{displaymath}
		\end{axiom}
		
		\begin{thm}
			Take any two inductive sets, $S$ and $T$. Then, $S \cap T$ is also an inductive set.
		\end{thm}
		
		\begin{proof}
			Let $U=S \cap T$.
			\begin{enumerate}
				\item $\emptyset \in U$
				\subitem $\emptyset \in S$ and $\emptyset \in T$ since $S$ and $T$ are both inductive.
				\item $(x \in U) \Rightarrow (\sigma(x) \in U)$
				\subitem $\forall x \in U, (x \in S) \wedge (x \in T)$.\\
				Since $S$ and $T$ are both inductive, $(\sigma(x) \in S) \wedge (\sigma(x) \in T)$\\
				Therefore $\sigma(x) \in U$.
			\end{enumerate} 
			Therefore $U$ is inductive.
		\end{proof}
		
		\begin{coro}
			An intersection of any number of inductive sets is inductive.
		\end{coro}
		
		\begin{thm}
			For any inductive set $S$, define $N_S$ as follows:
			\begin{displaymath}
			N_S=\bigcap_{\substack{A \subseteq S\\A \text{ is inductive}}}A
			\end{displaymath}
			Take any two inductive sets, $S$ and $T$. Then $N_S=N_T$.
		\end{thm}
		
		\begin{proof}
			Suppose not; WLOG, $\exists x$ such that $x \in N_S$ and $x \notin N_T$.\\
			Let $X=N_S \cap N_T$. Then $X$ is inductive, $X \subset N_S$, and $x \notin X$.\\
			Since by the definition of $N_S$, $N_S=X \cap N_S$, but $x \notin X \cap N_S$ hence the RHS and the LHS are different.\\
			Therefore the assumption is wrong; therefore $N_S=N_T$.
		\end{proof}
		
		Using this theorem, we can finally define the set of natural numbers:
		\begin{defn}[The Set $\mathbb(N)$ of natural numbers] \label{def_N}
			Take any inductive set $S$, and let
			\begin{displaymath}
			N=\bigcap_{\substack{A \subseteq S\\A \text{ is inductive}}}A
			\end{displaymath}
			This set is the natural numbers, which we denote as $\mathbb{N}$.
		\end{defn}
		
		\subsection{Operations on $\mathbb{N}$}
		We now define two operations on $mathbb{N}$, addition($+$) and multiplication($\cdot$).
		
		\begin{defn}[Addition and Multiplication on $\mathbb{N}$] \label{def_add_mult_N}
			The operation of addition, denoted by $+$, is defined by following two recursive rules:
			\begin{enumerate}
				\item $\forall n \in \mathbb{N}, n+0=n$
				\item $\forall n,m \in \mathbb{N}, n+\sigma(m)=\sigma(n+m)$
			\end{enumerate}
			Similarly the operation of multiplication, denoted by $\cdot$, is defined by following two recursive rules:
			\begin{enumerate}
				\item $\forall n \in \mathbb{N}, n \cdot 0=0$
				\item $\forall n,m \in \mathbb{N}, n \cdot \sigma(m)=n \cdot m+n$
			\end{enumerate}
		\end{defn}
		
		\begin{lemma}[Operations on $0$] \label{thm_n_op_on_0}
			$\forall x \in \mathbb{N}$
			\begin{itemize}
				\item $x+0=0+x$
				\item $x \cdot 0=0 \cdot x$
			\end{itemize}
		\end{lemma}
		
		\begin{prop}[Properties of $+$ and $\cdot$] \label{thm_property_of_operation_on_N}
			$\forall x,y,z \in \mathbb{N}$,
			\begin{itemize}
				\item \textbf{Associativity of Addition} $x+(y+z)=(x+y)+z$
				\item \textbf{Commutativity of Addition} $x+y=y+x$
				\item \textbf{Associativity of Multiplication} $x \cdot (y \cdot z)=(x \cdot y) \cdot z$
				\item \textbf{Commutativity of Multiplication} $x \cdot y=y \cdot x$
				\item \textbf{Distributive Law} $x \cdot (y+z)=x \cdot y+x \cdot z$
				\item \textbf{Cancellation Law for Addition} $x+z=y+z \Rightarrow x=y$
			\end{itemize}
		\end{prop}
		
		\subsection{Ordering on $\mathbb{N}$}
		\begin{defn}[Ordering on $\mathbb{N}$] \label{def_order_N}
			For $n,m \in \mathbb{N}$, we say that $n$ is less than $m$, written $n \ge m$, if there exists a $k \in \mathbb{N}$ such that $m=n+k$. We also write $n<m$ if $k \ne 0$.
		\end{defn}
		
		\begin{thm}
			$(N,<)$ is an ordered set[\ref{def_ordered_set}].
		\end{thm}
		
		\begin{prop}
			The followings are true:
			\begin{itemize}
				\item If $n \ne 0$, then $0<n$.
				\item Let $x,y,z \in \mathbb{N}$. Then the followings are true:
				\begin{itemize}
					\item $(x \le y) \wedge (y<z) \Rightarrow (x<z)$
					\item $(x<y) \wedge (y \le z) \Rightarrow (x<z)$
					\item $(x \le y) \wedge (y \le z) \Rightarrow (x \le z)$
					\item $(x<y) \Rightarrow (x+z<y+z)$
					\item $(x<y) \Rightarrow (xz<yz)$
				\end{itemize}
				\item $\forall n \in \mathbb{N}, n \ne n+1$
				\item $\forall n,k \in \mathbb{N}, k \ne 0, n \ne n+k$
			\end{itemize}
		\end{prop}
		
		\begin{defn}[Least Element] \label{def_least_element}
			Let $S \subset \mathbb{N}$. An element $n \in S$ is called a \emph{least element} if $\forall m \in S, n \le m$
		\end{defn}
		
		\begin{prop}[Uniqueness of the Least Element] \label{thm_unique_least_element}
			Let $S \subset \mathbb{N}$. Then if $S$ has a least element, then it is unique.
		\end{prop}
		
		\begin{thm}[Well-Ordering Property] \label{thm_well_ordering_property}
			Let $S$ be a nonempty subset of $\mathbb{N}$. Then $S$ has a least element.
		\end{thm}
		
		\begin{note}
			The well-ordering property states that the set of natural numbers $\mathbb{N}$ has the greatest lower bound property[\ref{def_greatest_lower_bound_property}] and thereby theorem \ref{thm_glb_lub_property}, has the least upper bound property[\ref{def_least_upper_bound_property}].
		\end{note}
		
		\subsection{Properties of $\mathbb{N}$}
		Many of the mathematics book defines the set of Natural Numbers as the set satisfying the \emph{Peano Axioms}.
		\begin{prop}[Peano Axioms] \label{peano_axioms}
			\begin{enumerate}
				\item[]
				\item $0$, which we defined as the empty set $\emptyset$, is a natural number.
				\item There exist a distinguished set map $\sigma: \mathbb{N} \rightarrow \mathbb{N}$
				\item $\sigma$ is injective
				\item There does not exist an element $n \in \mathbb{N}$ such that $\sigma(n)=0$
				\item (Principle of Induction) If $S \in N$ is inductive, then $S=N$.
			\end{enumerate}
		\end{prop}
		
		\begin{prop}
			Suppose that $a$ is a natural number, and that $b \in a$. Then $b \subseteq a$, $a \nsubseteq b$.
		\end{prop}
		
		\begin{prop}
			For any two natural numbers $a,b\in \mathbb{N}$, if $\sigma(a)=\sigma(b)$, then $a=b$.
		\end{prop}
		
		\begin{lemma}
			If $n \in \mathbb{N}$ and $n \ne 0$, then there exists $m \in \mathbb{N}$ such that $\sigma(m)=n$.
		\end{lemma}
	
	\section{$\mathbb{Z}$: The set of Integers}
		\subsection{Construction of $\mathbb{Z}$}
		\subsection{Operations on $\mathbb{Z}$}
		\subsection{Ordering on $\mathbb{Z}$}
		\subsection{Property of $\mathbb{Z}$}
	\section{$\mathbb{Q}$: The set of Rational Numbers}
		\subsection{Construction of $\mathbb{Q}$}
		\subsection{Operations on $\mathbb{Q}$}
		\subsection{Ordering on $\mathbb{Q}$}
		\subsection{Property of $\mathbb{Q}$}
	\section{$\mathbb{R}$: The set of Real Numbers}
		\subsection{Construction of $\mathbb{R}$}
		
			\begin{thm}[Existence of $\mathbb{R}$] \label{thm_existence_real_number}
				There exists an ordered field $\mathbb{R}$ containing $\mathbb{Q}$ as a subfield which has the least-upper-bound property.
			\end{thm}
			
			\begin{defn}[Extended Real Number System] \label{def_extended_real_number_system}
				The \emph{extended real number system}, denoted $\overline{\mathbb{R}}$, $[-\infty,\infty]$, or $\mathbb{R} \cup \{-\infty,\infty\}$, consists of the real field $\mathbb{R}$ and two symbols, $+\infty$ and $-\infty$. We preserve the original order in $\mathbb{R}$, and define $\forall x \in \mathbb{R}$,
				\begin{displaymath}
					-\infty<x<\infty
				\end{displaymath}
			\end{defn}

			\begin{remark} \label{remark_extended_real_number_system_not_field}
				The extended real number system does not form a field.
			\end{remark}

		\subsection{Operations on $\mathbb{R}$}
		\subsection{Ordering on $\mathbb{R}$}
		\subsection{Property of $\mathbb{R}$}
	
	\section{$\mathbb{I}$: The set of Complex Numbers}
		\subsection{Construction of $\mathbb{I}$}
		\subsection{Operations on $\mathbb{I}$}
		\subsection{Ordering on $\mathbb{I}$}
		\subsection{Property of $\mathbb{I}$}
\end{document}