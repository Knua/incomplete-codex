\documentclass{report}

\begin{document}
	\section{$\mathbb{N}$: The set of Natural Numbers}
		\subsection{Construction of $\mathbb{N}$}
		We start from the Axioms of Set and the following definitions:
		
		\begin{defn}[Successor] \label{defn_successor}
			For any set $x$, the \emph{successor of $x$}, denoted $\sigma(x)$, is defined as the following set:
			\begin{displaymath}
			\sigma(x)=x \cup \{x\}
			\end{displaymath}
		\end{defn}
		
		Let us define $0=\emptyset$, $1=\sigma(\emptyset)=\sigma(0)$. Using the definition of successors, and following the pattern, $2=\sigma(1)$, $3=\sigma(2)$, and so on. Basically we can make any finite number using the definition of successor and the Axioms of Set, but actually getting all of the natural numbers at once(or any infinitely large set, since only the empty set is guaranteed to exist by the axioms) is not possible with our axioms. We define the concept of Inductive Sets and make another Axiom for this purpose:
		
		\begin{defn}[Inductive Set] \label{defn_inductive_set}
			A set $A$ is called \emph{inductive} if it satisfies the following two properties:
			\begin{itemize}
				\item $\emptyset \in A$
				\item $(x \in A) \Rightarrow (\sigma(x) \in A)$
			\end{itemize}
		\end{defn}
		
		\begin{axiom}[Axiom of Infinity] \label{axiom_infinity}
			There is an inductive set, that is:
			\begin{displaymath}
			\exists A (\emptyset \in A) \wedge (\forall x \in A, \sigma(x) \in A)
			\end{displaymath}
		\end{axiom}
		
		\begin{thm}
			Take any two inductive sets, $S$ and $T$. Then, $S \cap T$ is also an inductive set.
		\end{thm}
		
		\begin{proof}
			Let $U=S \cap T$.
			\begin{enumerate}
				\item $\emptyset \in U$
				\subitem $\emptyset \in S$ and $\emptyset \in T$ since $S$ and $T$ are both inductive.
				\item $(x \in U) \Rightarrow (\sigma(x) \in U)$
				\subitem $\forall x \in U, (x \in S) \wedge (x \in T)$.\\
				Since $S$ and $T$ are both inductive, $(\sigma(x) \in S) \wedge (\sigma(x) \in T)$\\
				Therefore $\sigma(x) \in U$.
			\end{enumerate} 
			Therefore $U$ is inductive.
		\end{proof}
		
		\begin{coro}
			An intersection of any number of inductive sets is inductive.
		\end{coro}
		
		\begin{thm}
			For any inductive set $S$, define $N_S$ as follows:
			\begin{displaymath}
			N_S=\bigcap_{\substack{A \subseteq S\\A \text{ is inductive}}}A
			\end{displaymath}
			Take any two inductive sets, $S$ and $T$. Then $N_S=N_T$.
		\end{thm}
		
		\begin{proof}
			Suppose not; WLOG, $\exists x$ such that $x \in N_S$ and $x \notin N_T$.\\
			Let $X=N_S \cap N_T$. Then $X$ is inductive, $X \subset N_S$, and $x \notin X$.\\
			Since by the definition of $N_S$, $N_S=X \cap N_S$, but $x \notin X \cap N_S$ hence the RHS and the LHS are different.\\
			Therefore the assumption is wrong; therefore $N_S=N_T$.
		\end{proof}
		
		Using this theorem, we can finally define the set of natural numbers:
		\begin{defn}[The Set $\mathbb(N)$ of natural numbers] \label{def_N}
			Take any inductive set $S$, and let
			\begin{displaymath}
			N=\bigcap_{\substack{A \subseteq S\\A \text{ is inductive}}}A
			\end{displaymath}
			This set is the natural numbers, which we denote as $\mathbb{N}$.
		\end{defn}
		
		\subsection{Operations on $\mathbb{N}$}
		We now define two operations on $mathbb{N}$, addition($+$) and multiplication($\cdot$).
		
		\begin{defn}[Addition and Multiplication on $\mathbb{N}$] \label{def_add_mult_N}
			The operation of addition, denoted by $+$, is defined by following two recursive rules:
			\begin{enumerate}
				\item $\forall n \in \mathbb{N}, n+0=n$
				\item $\forall n,m \in \mathbb{N}, n+\sigma(m)=\sigma(n+m)$
			\end{enumerate}
			Similarly the operation of multiplication, denoted by $\cdot$, is defined by following two recursive rules:
			\begin{enumerate}
				\item $\forall n \in \mathbb{N}, n \cdot 0=0$
				\item $\forall n,m \in \mathbb{N}, n \cdot \sigma(m)=n \cdot m+n$
			\end{enumerate}
		\end{defn}
		
		\begin{lemma}[Operations on $0$] \label{thm_n_op_on_0}
			$\forall x \in \mathbb{N}$
			\begin{itemize}
				\item $x+0=0+x$
				\item $x \cdot 0=0 \cdot x$
			\end{itemize}
		\end{lemma}
		
		\begin{prop}[Properties of $+$ and $\cdot$] \label{thm_property_operation_N}
			$\forall x,y,z \in \mathbb{N}$,
			\begin{itemize}
				\item \textbf{Associativity of Addition} $x+(y+z)=(x+y)+z$
				\item \textbf{Commutativity of Addition} $x+y=y+x$
				\item \textbf{Associativity of Multiplication} $x \cdot (y \cdot z)=(x \cdot y) \cdot z$
				\item \textbf{Commutativity of Multiplication} $x \cdot y=y \cdot x$
				\item \textbf{Distributive Law} $x \cdot (y+z)=x \cdot y+x \cdot z$
				\item \textbf{Cancellation Law for Addition} $x+z=y+z \Rightarrow x=y$
			\end{itemize}
		\end{prop}
		
		\subsection{Ordering on $\mathbb{N}$}
		\begin{defn}[Ordering on $\mathbb{N}$] \label{def_order_N}
			For $n,m \in \mathbb{N}$, we say that $n$ is less than $m$, written $n \ge m$, if there exists a $k \in \mathbb{N}$ such that $m=n+k$. We also write $n<m$ if $k \ne 0$.
		\end{defn}
		
		\begin{thm}
			$(N,<)$ is an ordered set[\ref{def_ordered_set}].
		\end{thm}
		
		\begin{prop} \label{thm_property_order_N}
			The followings are true:
			\begin{itemize}
				\item If $n \ne 0$, then $0<n$.
				\item Let $x,y,z \in \mathbb{N}$. Then the followings are true:
				\begin{itemize}
					\item $(x \le y) \wedge (y<z) \Rightarrow (x<z)$
					\item $(x<y) \wedge (y \le z) \Rightarrow (x<z)$
					\item $(x \le y) \wedge (y \le z) \Rightarrow (x \le z)$
					\item $(x<y) \Rightarrow (x+z<y+z)$
					\item $(x<y) \Rightarrow (xz<yz)$
				\end{itemize}
				\item $\forall n \in \mathbb{N}, n \ne n+1$
				\item $\forall n,k \in \mathbb{N}, k \ne 0, n \ne n+k$
			\end{itemize}
		\end{prop}
		
		\begin{defn}[Least Element] \label{def_least_element}
			Let $S \subset \mathbb{N}$. An element $n \in S$ is called a \emph{least element} if $\forall m \in S, n \le m$
		\end{defn}
		
		\begin{prop}[Uniqueness of the Least Element] \label{thm_unique_least_element}
			Let $S \subset \mathbb{N}$. Then if $S$ has a least element, then it is unique.
		\end{prop}
		
		\begin{thm}[Well-Ordering Property] \label{thm_well_ordering_property}
			Let $S$ be a nonempty subset of $\mathbb{N}$. Then $S$ has a least element.
		\end{thm}
		
		\begin{note}
			The well-ordering property states that the set of natural numbers $\mathbb{N}$ has the greatest lower bound property[\ref{def_greatest_lower_bound_property}] and thereby theorem \ref{thm_glb_lub_property}, has the least upper bound property[\ref{def_least_upper_bound_property}].
		\end{note}
		
		\subsection{Properties of $\mathbb{N}$}
		Many of the mathematics book defines the set of Natural Numbers as the set satisfying the \emph{Peano Axioms}.
		\begin{prop}[Peano Axioms] \label{peano_axioms} \label{thm_property_N}
			\begin{enumerate}
				\item[]
				\item $0$, which we defined as the empty set $\emptyset$, is a natural number.
				\item There exist a distinguished set map $\sigma: \mathbb{N} \rightarrow \mathbb{N}$
				\item $\sigma$ is injective
				\item There does not exist an element $n \in \mathbb{N}$ such that $\sigma(n)=0$
				\item (Principle of Induction) If $S \in N$ is inductive, then $S=N$.
			\end{enumerate}
		\end{prop}
		
		\begin{prop}
			Suppose that $a$ is a natural number, and that $b \in a$. Then $b \subseteq a$, $a \nsubseteq b$.
		\end{prop}
		
		\begin{prop}
			For any two natural numbers $a,b\in \mathbb{N}$, if $\sigma(a)=\sigma(b)$, then $a=b$.
		\end{prop}
		
		\begin{lemma}
			If $n \in \mathbb{N}$ and $n \ne 0$, then there exists $m \in \mathbb{N}$ such that $\sigma(m)=n$.
		\end{lemma}
	
	\section{$\mathbb{Z}$: The set of Integers}
		\subsection{Construction of $\mathbb{Z}$}
		We now have the set of natural numbers, and starting there, we construct the set of integers.
		
		\begin{prop} \label{def_Z_equiv_class}
			Define a relation $\equiv$ on $\mathbb{N} \times \mathbb{N}$ by $(a,b) \equiv (c,d)$ iff $a+d=b+c$. This relation is an equivalence relation on $\mathbb{N} \times \mathbb{N}$.
		\end{prop}
	
		Let $\mathbb{Z}$ be the set of equivalence classes under this relation, and the equivalence class containing $(a,b)$ be denoted by $[a,b]$.
		
		\subsection{Operations on $\mathbb{Z}$}
		\begin{defn}[Addition and Multiplication on $\mathbb{Z}$] \label{def_add_mult_Z}
			Addition and multiplication on $\mathbb{Z}$ are defined by:
			\begin{itemize}
				\item $[a,b]+[c,d]=[a+c,b+d]$
				\item $[a,b] \cdot [c,d]=[ac+bd,ad+bc]$
			\end{itemize}
		\end{defn}
		
		\begin{defn}[Subtraction on $\mathbb{Z}$] \label{def_sub_Z}
			Subtraction on $\mathbb{Z}$ is defined by:
			\begin{displaymath}
				[a,b]-[c,d]=[a,b]+[d,c]
			\end{displaymath}
		\end{defn}
	
		\subsection{Ordering on $\mathbb{Z}$}

			\begin{defn}[Ordering on $\mathbb{Z}$] \label{def_order_Z}
				Let $[a,b],[c,d] \in \mathbb{Z}$. $[a,b]<[c,d]$ iff $a+d<b+c$.
			\end{defn}

\begin{comment}%todo: find out what the positive set is
			\begin{prop}
				The subset $\mathbb{N}=\{[0,n]|n \text{ is a natural number}\}$ is a positive set in $\mathbb{Z}$.
			\end{prop}
		
			\begin{proof}
				$\mathbb{N}$ is closed under addition and multiplication by definition.\\
				Let $[a,b] \in  \mathbb{Z}$. Since $a,b \in \mathbb{N}$ and $\mathbb{N}$ is well-ordered, either $a<b$, $a=b$, or $b<a$. So either $[a,b]$ is $[0,b-a]$, $[0,0]$, or $[a-b,0]$.\\
				Thus either $[a,b]\in\mathbb{N}$, $[a,b]=0$, or $-[a,b]=[b,a]\in\mathbb{n}$.
			\end{proof}
\end{comment}
	
		\subsection{Property of $\mathbb{Z}$}
		
			\begin{thm}[Arithmetic Properties of $\mathbb{Z}$] \label{thm_property_Z}
				\begin{enumerate}
					\item[]
					\item Addition and multiplication are well-defined.
					\item Addition and multiplication have identity elements $[n,n]$ and $[n,n+1]$, respectively.
					\item Addition and multiplication are commutative and associative.
					\item The distributive law holds.
					\item Each element $[a,b]$ has an additive inverse $[b,a]$.
				\end{enumerate}
			\end{thm}
	
			We can treat $\mathbb{N}$ to be a subset of $\mathbb{Z}$ by identifying the number $n$ with the class $[0,n]$. Since $[0,a]+[0,b]=[0,a+b]$ and $[0,a] \cdot [0,b]=[0,ab]$, these operations mirror the corresponding operation in $\mathbb{N}$.
			
			Given $n \in \mathbb{N}$, we write $-n$ for $[n,0]$, $0$ for $[n,n]$, and $1$ for $[n,n+1]$. By the fifth arithmetic property of $\mathbb{Z}$[\ref{thm_property_Z}], this defines $-n$ to be the additive inverse of $n$. We also use the minus sign for subtraction; it is therefore natural to write $[a,b]$ as $b-a$.
			
		\begin{prop}
			For $a,b \in \mathbb{N}$, let $-b$, $a$, and $b$ be defined in $\mathbb{Z}$ as above. Then
			\begin{displaymath}
				a-b=a+(-b) \text{   and   } -(-b)=b
			\end{displaymath}
		\end{prop}

	\section{$\mathbb{Q}$: The set of Rational Numbers}
		We construct the set of rational numbers from the set of integers as follows:
		\subsection{Construction of $\mathbb{Q}$}
		
			\begin{prop}\label{def_Q_equiv_class}
				Define a relation $\equiv$ on $\mathbb{Z} \times (\mathbb{Z} \textbackslash \{0\})$ by $(a,b)\equiv(c,d)$ iff $ad=bc$. This relation is an equivalence relation on $\mathbb{Z} \times (\mathbb{Z} \textbackslash \{0\})$.
			\end{prop}
		
			Let $\mathbb{Q}$ be the set of equivalence classes under this relation, and the equivalence class containing $(a,b)$ is denoted by $a/b$ or $\frac{a}{b}$, and $\frac{a}{b}=\frac{c}{d}$ mean that $(a,b)$ and $(c,d)$ belong to the same equivalence class. Especially we write $0$ and $1$ to denote $\frac{0}{1}$ and $\frac{1}{1}$, respectively.
		
		\subsection{Operations on $\mathbb{Q}$}
		
			\begin{defn}[Addition and Multiplication on $\mathbb{Q}$] \label{def_add_mult_Q}
				The \emph{sum} and \emph{product} of $\frac{a}{b}, \frac{c}{d} \in \mathbb{Q}$ are defined by
				\begin{displaymath}
					\frac{a}{b}+\frac{c}{d}=\frac{ad+bc}{bd} \text{   and   } \frac{a}{b}\frac{c}{d}=\frac{ac}{bd}
				\end{displaymath}
			\end{defn}
		
			\begin{defn}[Subtraction on $\mathbb{Q}$] \label{def_sub_Q}
				Subtraction on $\mathbb{Z}$ is defined by:
				\begin{displaymath}
					\frac{a}{b}-\frac{c}{d}=\frac{ad-bc}{bd}
				\end{displaymath}
			\end{defn}
			
			\begin{defn}[Division on $\mathbb{Q}$] \label{def_div_Q}
				Division on $\mathbb{Z}$ is defined by:
				\begin{displaymath}
					\frac{a}{b}\div\frac{c}{d}=\frac{ad}{bc}
				\end{displaymath}
			\end{defn}
		
		\subsection{Ordering on $\mathbb{Q}$}
		
			\begin{defn}[Ordering on $\mathbb{Q}$] \label{def_order_Q}
				Let $\frac{a}{b},\frac{c}{d} \in \mathbb{Q}$. $\frac{a}{b}<\frac{c}{d}$ iff $(bd>0 \wedge ad<bc) \vee (bd<0 \wedge ad>bc)$.
			\end{defn}
			
\begin{comment}%todo: find out what the positive set is
			\begin{prop}
				The subset $P=\{\frac{a}{b}\in\mathbb{Q}|ab>0}$ is a positive set in $\mathbb{Q}$.
			\end{prop}
\end{comment}
		
		\subsection{Property of $\mathbb{Q}$}
		
			\begin{thm}[Arithmetic Properties of $\mathbb{Q}$] \label{thm_property_Q}
				\begin{enumerate}
					\item[]
					\item Addition and multiplication are well-defined.
					\item Addition and multiplication have identity elements $0$ and $1$, respectively.
					\item Addition and multiplication are commutative and associative.
					\item The distributive law holds.
					\item Each element $\frac{a}{b}$ has an additive inverse $\frac{b}{a}$.
				\end{enumerate}
			\end{thm}
		
			\begin{thm}
				$(\mathbb{Q},+,\cdot)$ forms an ordered field.
			\end{thm}
		
	\section{$\mathbb{R}$: The set of Real Numbers}
		\subsection{Construction of $\mathbb{R}$}
			One simple way to construct $\mathbb{R}$ is by proving the following theorem:
			
			\begin{thm}[Existence of $\mathbb{R}$] \label{thm_existence_real_number}
				There exists an ordered field $\mathbb{R}$ containing $\mathbb{Q}$ as a subfield which has the least-upper-bound property.
			\end{thm}
		
			But where's the fun in that? We will be constructing the field of real numbers using Cauchy sequences[\ref{def_cauchy_sequence}], starting with the following proposition:
		
			\begin{thm}
				Define a relation $\equiv$ on the set $S$ of Cauchy sequences of rational numbers as follows:
				\begin{displaymath}
					\{a_n\} \equiv \{b_n\} \text{   iff   } (a_n-b_n)\rightarrow 0
				\end{displaymath}
				This relation is an equivalence relation.
			\end{thm}
		
			Now let us define $\mathbb{R}$ as the set of equivalence classes of $S$ under the relation $\equiv$.

		\subsection{Operations on $\mathbb{R}$}
		
			Before the definition of operations on $\mathbb{R}$, we need to find out whether if the Cauchy sequences of rational numbers are closed under addition and multiplication, and it turns out they do, as stated in the following proposition:
		
			\begin{prop}
				The set $S$ of Cauchy sequences of rational numbers is closed under addition, multiplication, and scalar multiplication, that is:
				\begin{enumerate}
					\item If $\{a_n\}\in S$ and $\{b_n\}\in S$, then $\{a_n+b_n\}\in S$
					\item If $\{a_n\}\in S$ and $\{b_n\}\in S$, then $\{a_nb_n\}\in S$
					\item If $\{a_n\}\in S$ and $c \in \mathbb{Q}$, then $\{ca_n\}\in S$
				\end{enumerate}
			\end{prop}
		
			We can finally go on to defining the operations on $\mathbb{R}$.
		
			\begin{defn}[Addition and Multiplication on $\mathbb{R}$] \label{def_add_mult_R}
				Let $\{a_n\}$ and $\{b_n\}$ be sequences contained in the real numbers $\alpha$, $\beta$, respectively. Then the \emph{sum} and \emph{product} of $\alpha$ and $\beta$ are defined by:
				\begin{displaymath}
					\alpha+\beta=\{a_n+b_n\} \text{   and   } \alpha\beta=\{\alpha\beta\}
				\end{displaymath}
			\end{defn}
		
			We can define subtraction and division on $\mathbb{R}$ similar to addition and multiplication, by term-by-term calculation on each term of the Cauchy sequence.
		
		\subsection{Ordering on $\mathbb{R}$}
		
			\begin{defn}[Ordering on $\mathbb{R}$] \label{def_order_R}
				Let $\alpha=\{a_n\},\beta=\{b_n\} \in \mathbb{R}$. $\alpha<\beta$ iff $\exists N \in \mathbb{N}, \forall n \ge N, a_n<b_n$.
			\end{defn}
		
		\subsection{Property of $\mathbb{R}$}
		
			\begin{thm}[Arithmetic Properties of $\mathbb{R}$] \label{thm_property_R}
				\begin{enumerate}
					\item[]
					\item Addition and multiplication are well-defined.
					\item Addition and multiplication have identity elements $\{0\}$ and $\{1\}$, respectively.
					\item Addition and multiplication are commutative and associative.
					\item The distributive law holds.
					\item Each element $\{a_n\}$ has an additive inverse $\{-a_n\}$.
				\end{enumerate}
			\end{thm}
		
			\begin{thm}
				$(\mathbb{R},+,\cdot)$ forms an ordered field.
			\end{thm}

			We now define an extension to $\mathbb{R}$ as follows:

			\begin{defn}[Extended Real Number System] \label{def_extended_real_number_system}
				The \emph{extended real number system}, denoted $\mathbb{R}^+$, $[-\infty,\infty]$, or $\mathbb{R} \cup \{-\infty,\infty\}$, consists of the real field $\mathbb{R}$ and two symbols, $+\infty$ and $-\infty$. We preserve the original order in $\mathbb{R}$, and define $\forall x \in \mathbb{R}$,
				\begin{displaymath}
				-\infty<x<\infty
				\end{displaymath}
			\end{defn}
			
			\begin{remark} \label{remark_extended_real_number_system_not_field}
				The extended real number system does not form a field.
			\end{remark}

	\section{$\mathbb{I}$: The set of Complex Numbers}
		\subsection{Construction of $\mathbb{I}$}
		
			
		
		\subsection{Operations on $\mathbb{I}$}
		
			
		
		\subsection{Ordering on $\mathbb{I}$}
		
			
		
		\subsection{Property of $\mathbb{I}$}
		
			
		
\end{document}